\Section{Introduction}

Move \cite{MOVE_LANG} is a new high-level programming language for writing smart
contracts.  Move and it's underlying engine, the Move Virtual Machine, have been
designed with \emph{formal verification} in mind. Specification related language
support is integrated into the Move language, and the \emph{Move prover} tool
has been developed, allowing to verify specifications against the implementation.

In this paper, we describe for the first time the methodology and theory behind
\emph{practical formal verification} in Move.  Since the earlier publication
about the Move prover tool in \cite{MOVE_PROVER}, many changes have been made to
the Move specification language and the prover implementation. Those changes
went hand-in-hand with the evolution of the \emph{Diem
  framework}~\cite{DIEM_FRAMEWORK}, which is a Move library for smart contracts
running on the Diem blockchain~\cite{DIEM}. The framework provides functionality
for managing accounts and their interaction, including multiple currencies,
account roles, and rules for transactions.  It consists of approximately 12,000
lines of Move program code and specifications.  The framework is exhaustively
specified, and \emph{verification runs fully automated alongside with unit and
  integration tests}, being integrated into the regular development process of
Diem.

The cornerstones for the success of this project are seen in the following
aspects.  First, the Move language was designed with verification in mind,
making it amenable for mechanized proof. Specifically its memory safety
properties and borrow semantics contribute to this. Second, the fully sandboxed
execution model of Move reduces the problem of outcalls to unspecified,
open-ended code which many verification approaches face. Third, the state of the
art in SMT tools like Z3~\cite{Z3} give rise to powerful practical
applications. Last but not least, the co-development of the Diem framework with
the specification language and prover provided a valuable feedback loop for
continuous improvement.

While those results look promising, constituting one of the larger recent
applications of formal verification in industry, the technology is not yet fully
ready for mainstream usage. A major classical obstacle of SMT-based verification
remains: as we are dealing with undecidable problems, heuristics in the solver
can fail, leading to occasional verification timeouts, which require support of
specialized engineers to solve. Also, specifications are arguably harder to
write than code, and require significant effort. Moreover, diagnosis produced by
the prover on verification failures, while already better than one would expect,
need to be further improved.  We describe the obstacles for mainstream
usability, and our ideas how they might be overcome in the conclusion of this
paper.

\Paragraph{Acknowledgement} Many more people have contributed to the Move Prover: Sam Blackshear, Mathieu Baudet, Todd Nowacki, Bob Wilson, Tim Zaikan, \ldots (list interns and other Move team members)



%%% Local Variables:
%%% mode: latex
%%% TeX-master: "main"
%%% End:
