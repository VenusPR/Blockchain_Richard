\Section{Overview of Move}

Move was developed for the Diem blockchain \cite{DIEM}, but its design is not
specific to blockchains.  A Move execution consists of a sequence of updates
evolving a \emph{global persistent memory state}, which we just call the
\emph{(global) memory}.  Updates are executed in a \emph{transactional} style:
the next memory state they compute will only be committed if their computation
has finished successfully and the result can be merged back without conflict
into the current memory state. Semantically, a Move execution can therefore be
interpreted as a labelled transition system (i.e. interleaved execution
steps). Any state evolving system which is adequately modeled by this semantics,
not just blockchains, can be programmed in Move (for example, transactions on a
concurrent data base, or training steps in an iterative ML algorithm).

The Move language allows to define memory in terms of so-called
\emph{resources}.  Resources are data structures which are stored in memory
indexed by \emph{account addresses}. For example, the Move expression
|exists<Balance>(account)| determines whether the resource |Balance| exists at
address |account|. As resource types can be generic (for example,
|Balance<Currency>|), an index for a resource is a tuple of types and the
account address: semantically, for each resource type $R$ there is a partial
memory function $\smem_R \in \sseq{\stype} \scross \saddr \sinto \swbot{R}$,
with $\sseq{\stype}$ a sequence of types use to instantiate
$R\sgen{\sseq{\stype}}$, and $\saddr$ the domain of addresses. Notice that
account addresses are not just arbitrary values but have a specific role in
Move's programming methodology related to access control via the builtin type of
\emph{signers}, as will be discussed later.

A Move application consists of a set of \emph{transaction scripts}. Each of
those script defines a Move function with input parameters but no output
parameters.  This function updates the memory $\smem$. The execution of this
function can fail via a well-defined abort mechanism, in which case $\smem$
stays unmodified. An environment emits a sequence of calls to such scripts,
thereby evolving $\smem$. To understand this execution as an LTS, consider the
set of states to be $\smem$, the labels the names of transaction scripts
combined with a set of concrete parameters, and the transition relation defined
by the transaction scripts. Abortion of the transaction function creates a label
with the script name, the parameters, and information about the abort reason,
which cycles on the current state.

A transaction script is written in Move as an imperative function which can read
and write the global memory $\smem$. Move uses a specific style of
imperative programming based on \emph{borrow semantics} \cite{BORROW_SEM}, as
popularized in the programming language Rust \cite{RUST,RUST_TYPES}. For the
verification problem borrow semantics is very important.  While allowing
references into structured data, those are guaranteed to be safe by the
\emph{borrow checker} \cite{BORROW_CHECKER}, which is run during bytecode
loading time, and which verification can assume. Furthermore, the notorious hard
verification problem of aliasing of references in the presence of mutation is
eliminated.  Mutation always starts from a root location either in global memory
or on the execution stack, and while a tree starting from this root is mutated,
no other access can happen anywhere in the tree. Intuitively, borrow semantics
allows to move a mutation 'cursor' down the tree, which follows linear typing
discipline. Because of this property, mutable reference parameters to functions
can be converted to input/output parameters, and verification of Move can avoid
the traditionally hard problems caused by aliasing of mutable references.


\SubSection{Programming in Move}

In Move, one defines transactions via so-called \emph{script
  functions} which take a set of parameters.  Those functions can call other
functions. Script and regular functions are encapsulated in \emph{modules}. Move
modules are also the place where resource types and other structured data is
defined.


We illustrate the language by example (for a more complete description of Move,
see the online documentation~\cite{MOVE_LANG_DEF}). The example is a simple
account which holds a balance of coins and is given in
Fig.~\ref{fig:AccountDef}. The transaction is to transfer coins from one to
another account~\footnote{Indeed, for a complete system, transactions like
  creating an account and funding it would be needed, but we leave this aspect
  out here.}.


\begin{figure}[t!]
\caption{\label{fig:AccountDef} Account Example Program}
\begin{MoveBox}
module Account {
  struct Coin has store {
    value: u64
  }
  struct Account has key {
    balance: Coin,
  }

  public fun withdraw(account: address, amount: u64): Coin
  acquires Account {
    let balance = &mut borrow_global_mut<Account>(account).balance;
    assert(balance.value >= amount, Errors::limit_exceeded());
    balance.value = balance.value - amount;
    Coin{value: amount}
  }

  public fun deposit(account: address, check: Coin)
  acquires Account {
    let Coin{value: amount} = check; // Consume coin   @\label{line:DropCoin}@
    let balance = &mut borrow_global_mut<Account>(account).balance;
    assert(balance.value <= Limits::max_u64() - amount,
           Errors::limit_exceeded());
    balance.value = balance.value + amount;
  }

  public(script) fun transfer(from: &signer, to: address, amount: u64) {
    let coin = Account::withdraw(Signer::address_of(from), amount);
    Account::deposit(to, move(coin)) @\label{line:CoinMove}@
  }
}
\end{MoveBox}
\end{figure}


\begin{itemize}
\item The struct |Account::Coin| represents units of currency. The |has store|
  ability of the |Coin| struct indicates that it can be stored as a field in
  another struct. Notice that by default, in Move, structs have \emph{linear}
  semantics: a |Coin| cannot be copied and dropped without explict destruction
  (as on line~\ref{line:DropCoin}). This is useful to prevent accidental
  duplication or lost of coins. To indicate that the struct can be copied and
  dropped, one would need to add the abilities |has copy, drop|.
\item |Coin| is aggregated in the struct |Account| for representing a
  \emph{balance}; the ability |has key| indicates that this struct can be stored
  as a resource in global memory.
\item The |Account::withdraw| function subtracts a value from the balance,
  returning a new |Coin| for the withdrawn amount.  It uses the builtin function
  |borrow_global_mut<T>(address)| which returns a mutable reference to the
  |Account| resource.  Similarily, |Account::deposit| takes a coin which is
  destructed and its amount added to the account.
\item The |acquire Account| modifier on a function declaration indicates that
  the function will borrow the |Account| global memory as a whole -- i.e. for
  every account address. The Move borrow checker will reject a call to such
  functions if any account resources are already borrowed, implementing memory
  safety for Move~\cite{BORROW_CHECKER}.
\item The |assert| statement causes a Move transaction to abort execution if the
  condition is not met, with the specified error code. No effects on the memory
  occur on abort. Abortion can also happen implicitly; for example, the
  expression |borrow_global_mut<T>(addr)| will abort if no resource |T| exists
  at |addr|.
\item The script |Account::transfer| is a top-level entry point into this Move
  program, calling |Account::withdraw| and |Account::deposit|. The call to the
  builtin function |move| at line~\ref{line:CoinMove} illustrates how the linear
  coin value travels from one call to another.
\item Scripts get passed in so called \emph{signers} which are tokens which
  represent an authorized account address. The caller of the script -- an
  external program -- has ensured that the owner of the signer account address has
  agreed to execute this transaction.
\end{itemize}

\SubSection{Specifying in Move}

The specification language supports {\em Design By Contract}
\cite{DESIGN_BY_CONTRACT}. Developers can provide pre and post conditions for
functions, which include conditions over (mutable) parameters and global
memory. Developers can also provide invariants over data structures, as well as
the (state-dependent) content of the global memory. Universal and
existential quantification both over bounded domains (like the indices of a
vector) as well of unbounded domains (like all memory addresses, all integers,
etc.)  are supported. The latter makes the specification language very
expressive, but also renders the verification problem in theory undecidable (and
in practice dependent on heuristic decision procedures).

\begin{figure}[t!]
\caption{\label{fig:AccountSpec} Account Example Specification}
\begin{MoveBox}
module Account {
  spec withdraw {
    aborts_if bal(account) < amount;
    ensures bal(account) == old(bal(account)) - amount;
    ensures result == Coin{value: amount};
    modifies global<Account>(acc); @\label{line:AccountModifies}@
  }

  spec deposit {
    aborts_if bal(account) + check.value > Limits::max_u64();
    ensures bal(account) == old(bal(account)) + check.value;
    modifies global<Account>(acc);
  }

  spec fun bal(acc: address): num { @\label{line:BalDef}@
    global<Account>(acc).balance.value
  }

  invariant forall acc: address where exists<Account>(acc): @\label{line:BalanceInvariant}@
    bal(acc) >= AccountLimits::MIN_BALANCE;

  invariant update forall acc: address where exists<Account>(acc): @\label{line:DecreaseInvariant}@
    old(bal(acc)) - bal(acc) <= AccountLimits::MAX_DECREASE;
}
\end{MoveBox}
\end{figure}

Fig.~\ref{fig:AccountSpec} illustrates the specification language by extending
the account example in Fig.~\ref{fig:AccountDef} (for the definition of the
specification language see \cite{MOVE_SPEC_LANG_DEF}).
\begin{itemize}
\item The function specification blocks |spec withdraw| and |spec desposit|
  specify when those functions abort, the expected effect on the global memory,
  and its return value (the return value is represented by the well-known name
  |result|).
\item As common in this style of specifications, in the |ensures| statement, by
  default the post-state of the function is referred to, whereas the form
  |old(..)| can be used to access the pre-state.
\item We are using the helper function |bal(address)| defined on
  line~\ref{line:BalDef} to access the value of the account balance. Helper
  functions can access state and can be transparently used within |old(..)|; the
  function is then evaluated in the pre-state.
\item The |modifies| statement on line~\ref{line:AccountModifies} specifies that
  this function only changes the indicated memory but no other memory.
\item The specification contains two invariants over global memory. The first
  invariant on line~\ref{line:BalanceInvariant} states that a balance can never
  drop underneath a certain minimum. The second invariant on
  line~\ref{line:DecreaseInvariant} refers to an update of global memory with
  pre and post state: the balance on an account can never decrease in one step
  more than a certain amount.
\item Note that while the Move programming language has only unsigned integers,
  the specification language uses arbitrary precision signed integers, making it
  convenient to specify something like |x - y <= limit|, without need to worry
  about underflow or overflow.
\end{itemize}

A discerning reader may have noted that the program in Fig.~\ref{fig:AccountDef}
does not actually satisfy the specification in Fig.~\ref{fig:AccountSpec}. This
will be discussed in the next section.

The constructs we have seen so far are only a subset of the available features
of the Move specification language. Notably, the language supports the following
additional features:

\begin{itemize}
\item Function preconditions via the |requires|-clause.
\item Data invariants for |struct| types, as a predicate over the field values.
\item Means to abstract commonly used specification fragments in so-called
  \emph{specification schemas} which can then be included in other specification
  blocks.
\end{itemize}

\SubSection{Running the Prover}
\label{sec:RunningProver}

The Move prover is a tool which supports verification of specifications as shown
above. The prover operates fully automated, quite similar as a type checker or
linter, and is expected to conclude in reasonable execution time, so it can be
integrated in the regular development workflow.

Running the prover on the program and specification of |module Account| produces
multiple errors, as mentioned. The first is this one:

\TODO{wrwg}{make line number symbolic so they align with figures}

\begin{MoveDiag}
error: abort not covered by any of the `aborts_if` clauses

    --- account.move:15:3 ---
    |
 15 |  public fun withdraw(account: address, amount: u64): Coin
    .
 18 |         &mut borrow_global_mut<Account>(account).balance;
    .              ----------------- abort happened here
    |
    =     at account.move:15:3: withdraw
    =         account = 0x19, amount = 15724
    =     at account.move:18:14: withdraw (ABORTED)
\end{MoveDiag}

\noindent The prover has detected that an implicit aborts condition is missing
in the specification of the |withdraw| function. It prints the context of the
error, as well as an \emph{execution trace} which lead to the error. Values of
variable assignments from the counter example found by the prover are printed
together with the execution trace. Logically, the counter example presents an
instance of assignments to variables such that program and specification
disagree. In general, the Move prover attempts to produce diagnostics readable
for Move developers without the need of understanding any internals of the
prover.


The next errors produced are about the memory invariants in
Fig.~\ref{fig:AccountSpec}. Both of them do not hold:

\begin{MoveDiag}
error: global memory invariant does not hold

    --- account.move:43:5 ---
    |
 43 | invariant forall acc: address where exists<Account>(acc):
 44 |     bal(acc) >= AccountLimits::MIN_BALANCE;
    |
    .
    =     at account.move:21:35: withdraw

error: global memory invariant does not hold

    --- account.move:45:5 ---
    |
 45 | invariant update
 46 |   forall acc: address where exists<Account>(acc):
 47 |     old(bal(acc)) - bal(acc) <= AccountLimits::MAX_DECREASE;
    .
    =     at account.move:21:35: withdraw
\end{MoveDiag}

\noindent This happens because in the program in Fig.~\ref{fig:AccountDef}, we
did not made any attempt to respect the limits in |MIN_BALANCE| and
|MAX_DECREASE|. We leave it open here how to fix this problem, which would
require to add some more |assert| statements to the code and abort if the limits
are not met.

%%% Local Variables:
%%% mode: latex
%%% TeX-master: "main"
%%% End:
