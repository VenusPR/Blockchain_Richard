\Section{Analysis}

\Paragraph{Predictability and Performance Improvements}

Compared to the version of the Move Prover which was released in September 2020
as part of version 1.0 of the Diem project we observe many improvements in
verification speed and predictability of verification outcome, which we track
back to the bytecode transformations we have described here. These results are,
however, not so easy to quantify, as the Move language has evolved and sources
written today are not compatible with sources this time back. Moreover, the Diem
framework and its specification, which is the basis of our benchmarking, has
evolved as well, adding more functionality and tighter specifications.

As one data point we use the |DiemAccount| module, the biggest module in the
Diem framework. This module encapsulates basic functionality to create and
maintain multiple types of accounts on the blockchain, as well as manage their
coin balances. It was called |LibraAccount| in release 1.0 of \MVP. The below
table lists the number of lines, functions, invariants, conditions (requires, ensures,
and aborts-if), as well as the verification times:

{
\setlength{\tabcolsep}{6pt}
\vspace{2ex}
\begin{tabular*}{0.9\textwidth}{cccccc}
  \hline
  \hline
  Module & Lines & Functions & Invariants & Conditions & Timing \\
  \hline
  LibraAccount & 1975 & 72 & 10 & 113 & \textbf{9.131s} \\
  DiemAccount & 2554 & 64 & 32 & 171 & \textbf{6.290s} \\
  \hline
\end{tabular*}
\vspace{2ex}
}

\noindent Notice that |DiemAccount| has significantly grown in size compared to
|LibraAccount|.  Specifically, additional specifications have been
added. Moreover, in the original |LibraAccount|, some of the most complex
functions had to be disabled for verification because the old version of \MVP
would timeout on them. In contrast, in |DiemAccount| and with the new version,
all functions are verified. Nevertheless, verification time has been improved by
roughly 30\%, in the presence of 3x more global invariants, and .5x more
function conditions.

We were able to observe similar improvements for the remaining of the 40 modules
of the Diem framework. All of the roughly 1/2 dozen timeouts in verification in
the framework resolved after introduction of the transformations described in
this paper. Also, specifications which were introduced after the new
transformations did not introduce new timeouts. A further improvement is that
often, in cases where specification and program disagree, a timeout occurred
which went away only after fixing the spec or the code, making debugging of such
verification failures rather hard. This problem also disappeared.


\Paragraph{Causes for the Improvements}

It is hard to clearly identify the reasons for improvements when dealing with
heuristic systems like SMT solvers. Moreover, besides of the transformations
described in this paper, many more smaller changes had been made to \MVP,
including micro-tuning the SMT encoding of the verification problem in
Boogie. Tracking the causes of improvements incrementally was not the highest
priority of our work.

Nevertheless, we believe that one of the biggest impacts, specifically regards
removal of timeouts and predictability of verification, is monomorphization. The
reason for this is that monomorphization allows a multi-sorted representation
of values in Boogie (and eventually the SMT solver). In contrast, before
monomorphization, we used a universal domain for values in order to represent
values in generic functions, roughly as follows:

\begin{Move}
  type Value = Num(int) | Address(int) | Struct(Vector<Value>) | ...
\end{Move}

\noindent This creates a large overhead for the SMT solver, as we need to
exhaustively inject type assumptions (e.g. that a |Value| is actually an
|Address|), and pack/unpack values. Consider a quantifier like~%
|forall a: address: P(x)| in Move. Before monomorphization, we have to represent
this in Boogie as~%
|forall a: Value: is#Address(a) => P(v#Address(a))|. This quantifier is
triggered where ever |is#Address(a)| is present, independent of the structure of
|P|. Over-triggering or inadequate triggering of quantifiers is one of the
suspected sources of timeouts, as also discussed in~\cite{BUTTERFLY}.

Moreover, before monomorphization, global memory was indexed in Boogie by an
address and a type instantiation. That is, for |struct R<T>| we would
have one Boogie array |[Type, int]Value|. With monomorphization, the type index
is eliminated, as we create different memory variables for each type
instantiation.  Quantification over memory content works now on a one-dimensional
instead of an n-dimensional Boogie array.

\Paragraph{Discussion and Related Work}

A recent survey of formal verification applied to smart contracts is found
in~\cite{CONTRACT_VERIFICATION}. The authors distinguish between \emph{contract}
and \emph{program} level approaches. Our approach has aspects of both: we
address program level properties via pre/post conditions, and contract
(``blockchain state'') level properties via global invariants. In both cases, we
use traditional predicate logic to write these properties, characterized as
Hoare logic by the paper.

While~\cite{CONTRACT_VERIFICATION} refers to at least two dozen systems for
smart contract verification, to the best of our knowledge, the Move ecosystem is
the first one where contract programming and specification language are fully
integrated, and the language is designed from first principles influenced by
verification. Methodologically, Move and the Move prover are therefore closer to
systems like Dafny~\cite{DAFNY}, or the older Spec\# system~\cite{SPECSHARP},
where instead of adding a specification approach posterior to an existing
language, it is part from the beginning. This allows us not only to deliver a
more consistent user experience, but also to make verification technically
easier by curating the programming language, as reflected in Move's absence of
dynamic dispatching and the notorious re-entrance problem~\cite{REENTRANCE}, as
well as the borrow semantics which enables optimizations like reference
elimination (Sec.~\ref{sec:RefElim}).

As regards expressiveness of specification, to the best of our knowledge, no
existing specification approach for smart contracts based on inductive Hoare
logic has similar expressiveness. We support universal quantification over
arbitrary memory content, a suspension mechanism of invariants to allow
non-atomic construction of memory content, and generic invariants. The standard
verification mechanism in Solidity does not support quantifiers, because it
interprets programming language constructs a specifications and has no
dedidacted specification language. While in Solidity one can simulate aspects of
global invariants using modifiers by attaching pre/post conditions, this is not
the same as our invariants, which are guaranteed to hold independent of whether
a user may or (accidentally) may not attach a modifier.  Moreover, from
experimenting with a similar approach in Move, we know that adding invariants as
pre/post conditions can be highly inefficient, because they need to be verified
independent from whether a function actually changes state. In contrast, our
approach to inject invariants optimizes when an invariant is actually verified.
While the expressiveness of Move specifications and \MVP comes with the price of
undecidability and the dependency from heuristics in SMT solvers, \MVP is
capable to deal with this by its elaborated translation to SMT logic, as
partially described in this paper. The result is a practical verification system
that is fully integrated into the Diem blockchain production process, which (to
the best of our knowledge) is novel by itself.

Borrow semantics is considered a good way to perform high-performance \emph{and}
safe programming in the Rust community. Even though the kind of reference
elimination we perform could also be done for the safe Rust language subset, to
the best of our knowledge, this has not been attempted before. The same
technique could likely not only be used for verification, but also for runtime
execution, potentially obtaining higher speed for smaller data structures by
improving processor cache locality.

\Paragraph{Open Problems and Future Work}

While currently \MVP is operating to our satisfaction, there are multiple open
problems as well. For one, the set of smart contracts we are verifying still
have only medium complexity, and specifically loops are rare in the Diem
framework (loops were not discussed in this paper for reasons of space; we do
incorporate loop invariants for their verification). We anticipate the problem
of timeouts to hit us again down the road as the number of applications of \MVP
grow, and further refinement of the translation will be needed. Furthermore,
while currently there are no occurrences of false positives, we expect this problem
to hit as well.

The biggest obstacle for wider adoption, however, is seen in the complexity of
authoring and reading specifications, related to a phenomena we refer to as
``specification boilerplate''. Pre/post condition specifications are often
verbose and difficult to read and write. At the time of this writing, function
specifications in the Diem framework are often of similar size than function
implementations.  A common problem is also sharing between aspects of specifications:
for example, if function |f| calls function |g| and the later function aborts, the
aborts condition propagates from |g| to the caller |f|. While the Move specification
language has mechanisms to express this kind of sharing by referring to the aborts
condition of |g| in the specification of |f|, this mechanism also makes specifications
less readable, and leaks implementation details into the specification of |f|.
We are currently experimenting with multiple strategies to cope with this, ranging
from a better methodology and language support to structure specifications, to
automatically inferring specs for some functions.



%%% Local Variables:
%%% mode: latex
%%% TeX-master: "main"
%%% End:
