\Section{Injection of Function Specifications}

During function specification injection, move specifications are reduced to basic
assume/assert statements added to the Move code.  Those statements represent
instructions to the solver backend about what propositions can be assumed and
which need to be asserted (verified) at a given program point.  In this appendix,
we cover how \emph{function specification conditions} are injected.

\SubSection{Pre- and Post conditions}

The injection of basic function specifications is illustrated in
Fig.~\ref{fig:RequiresEnsuresAbortsIf}.  An extension of the Move source
language is used to specify abort behavior. With~%
|fun f() { .. } onabort { conditions }| a Move function is defined where
|conditions| are assume or assert statements that are evaluated at every program
point the function aborts (either implicitly or with an |abort| statement). This
construct simplifies the presentation and corresponds to a per-function abort
block on bytecode level which is target of branching.

\begin{Figure}
  \caption{Requires, Ensures, and AbortsIf Injection}
  \label{fig:RequiresEnsuresAbortsIf}
  \centering
\begin{MoveBoxNumbered}
  fun f(x: u64, y: u64): u64 { x + y }
  spec f {
    requires x < y;
    aborts_if x + y > MAX_U64;
    ensures result == x + y;
  }
  fun g(x: u64): u64 { f(x, x + 1) }
  spec g {
    ensures result > x;
  }
  @\transform@
  fun f(x: u64, y: u64): u64 {
    spec assume x < y;
    let result = x + y;
    spec assert result == x + y;     // ensures of of
    spec assert                      // negated abort_if of f
      !(x + y > MAX_U64); @\label{line:aborts_holds_not}@
    result
  } onabort {
    spec assert                      // abort_if of f
      x + y > MAX_U64; @\label{line:aborts_holds}@
  }
  fun g(x: u64): u64 {
    spec assert x < x + 1;           // requires of f
@$\textrm{\it if inlined}$\label{line:inline}@
    let result = inline f(x, x + 1);
@$\textrm{\it elif opaque}$\label{line:opaque}@
    if (x + x + 1 > MAX_U64) abort;  // aborts_if of f
    spec assume result == x + x + 1; // ensures of f
@$\textrm{\it endif}$@
    spec assert result > x;          // ensures of g
    result
  }
\end{MoveBoxNumbered}
\end{Figure}

An aborts condition is translated into two different asserts: one where the
function aborts and the condition must hold (line~\ref{line:aborts_holds}), and
one where it returns and the condition must \emph{not} hold
(line~\ref{line:aborts_holds_not}). If there are multiple |aborts_if|, they are
or-ed. If there is no abort condition, no asserts are generated. This means
that once a user specifies aborts conditions, they must completely cover the
abort behavior of the code. (The prover also provides an option to relax this
behavior, where aborts conditions can be partial and are only enforced on
function return.)

For a function call site we distinguish two variants: the call is \emph{inlined}
(line~\ref{line:inline}) or it is \emph{opaque} (line~\ref{line:opaque}).  For
inlined calls, the function definition, with all injected assumptions and
assertions turned into assumptions (as those are considered proven) is
substituted. For opaque functions the specification conditions are inserted as
assumptions. Methodologically, opaque functions need precise specifications
relative to a particular objective, where as in the case of inlined functions
the code is still the source of truth and specifications can be partial or
omitted. However, inlining does not scale arbitrarily, and can be only used for
small function systems.

Notice we have not discussed the way how to deal with relating pre and post
states yet, which requires taking snapshots of state (e.g.~%
|ensures x == old(x) + 1|); the example in
Fig.~\ref{fig:RequiresEnsuresAbortsIf} does not need it. Snapshots of state
will be discussed for global update invariants in Sec.~\ref{sec:GlobalInvariants}.

\Paragraph{Modifies}

\begin{Figure}
  \caption{Modifies Injection}
  \label{fig:Modifies}
  \centering
\begin{MoveBoxNumbered}
  fun f(addr: address) { move_to<T>(addr, T{}) }
  spec f {
    pragma opaque;
    ensures exists<T>(addr);
    modifies global<T>(addr);
  }
  fun g() { f(0x1) }
  spec g {
    modifies global<T>(0x1); modifies global<T>(0x2);
  }
  @\transform@
  fun f(addr: address) {
    let can_modify_T = {addr};      // modifies of f
    spec assert addr in can_modify; // permission check @%
                                            \label{line:modifies_permission}@
    move_to<T>(addr, T{});
  }
  fun g() {
    let can_modify_T = {0x1, 0x2};  // modifies of g
    spec assert {0x1} <= can_modify_T; // permission check @%
                                            \label{line:modifies_call_permission}@
    spec havoc global<T>(0x1);      // havoc modified memory @%
                                            \label{line:modifies_havoc}@
    spec assume exists<T>(0x1);     // ensures of f
  }
\end{MoveBoxNumbered}
\end{Figure}


The |modifies| condition specifies that a function only changes specific memory.
It comes in the form |modifies global<T>(addr)|, and its injection is
illustrated in Fig.~\ref{fig:Modifies}.

A type check is used to ensure that if a function has one or more~%
|modifies| conditions all called functions which are \emph{opaque} have a matching
modifies declaration. This is important so we can relate the callees
memory modifications to that what is allowed at caller side.

At verification time, when an operation is performed which modifies memory, an
assertion is emitted that modification is allowed
(e.g. line~\ref{line:modifies_permission}). The permitted addresses derived from
the modifies clause are stored in a set |can_modify_T| generated by the
transformation. Instructions which modify memory are either primitives (like
|move_to| in the example) or function calls. If the function call is inlined,
modifies injection proceeds (conceptually) with the inlined body. For opaque
function calls, the static analysis has ensured that the target has a modifies
clause.  This clause is used to derive the modified memory, which must be a
subset of the modified memory of the caller
(line~\ref{line:modifies_call_permission}).

For opaque calls, we also need to \emph{havoc} the memory they modify
(line~\ref{line:modifies_havoc}), by which is meant assigning an unconstrained
value to it. If present, |ensures| from the called function, injected as
subsequent assumptions, are further constraining the modified memory.


\SubSection{Data Invariants}

\begin{Figure}
  \caption{Data Invariant Injection}
  \label{fig:DataInvariants}
  \centering
\begin{MoveBox}
  struct S { a: u64, b: u64 }
  spec S { invariant a < b }
  fun f(s: S): S {
    let r = &mut s;
    r.a = r.a + 1;
    r.b = r.b + 1;
    s
  }
  @\transform@
  fun f(s: S): S {
    spec assume s.a < s.b;      // assume invariant for s
    let r = Mvp::local(s, F_s); // begin mutation of s
    r = Mvp::set(r, Mvp::get(r)[a = Mvp::get(r).a + 1]);
    r = Mvp::set(r, Mvp::get(r)[b = Mvp::get(r).b + 1]);
    spec assert                 // invariant enforced
      Mvp::get(r).a < Mvp::get(r).b;
    s = Mvp::get(r);            // write back to s
    s
  }
\end{MoveBox}
\end{Figure}

A data invariant specifies a constraint over a struct value. The value is
guaranteed to satisfy this constraint at any time. Thus, when a value is
constructed, the data invariant needs to be verified, and when it is consumed,
it can be assumed to hold.

In Move's reference semantics, construction of struct values is often done via a
sequence of mutations via mutable references. It is desirable that \emph{during}
such mutations, assertion of the data invariant is suspended. This allows to
state invariants which reference multiple fields, where the fields are updated
step-by-step.  Move's borrow semantics and concept of mutations provides a
natural way how to defer invariant evaluation: at the point a mutable reference
is released, mutation ends, and the data invariant can be enforced.  In other
specification formalisms, we would need a special language construct for
invariant suspension. Fig.~\ref{fig:DataInvariants} gives an example, and shows
how data invariants are reduced to assert/assume statements.

\Paragraph{Implementation}

The implementation hooks into the reference elimination
(Sec.~\ref{sec:RefElim}). As part of this the lifetime of references is
computed. Whenever a reference is released and the mutated value is written
back, we also enforce the data invariant. In addition, the data invariant is
enforced when a struct value is directly constructed.




%%% Local Variables:
%%% mode: latex
%%% TeX-master: "main"
%%% End:
