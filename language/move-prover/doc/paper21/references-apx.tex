\section{Elimination of Mutable References with Dynamic Value}
\label{sec:RefElimApx}

In Sec.~\ref{sec:RefElimMut} we described how mutable references are eliminated
in the case their root location is statically known. This covers the majority of
typical Move code.  However, there are also important cases where a root location
depends on runtime decisions:

\begin{Move}
  let r = if (p) &mut s1 else &mut s2;
  increment_field(r);
\end{Move}

\noindent Additional information in the logical encoding is required to deal
with such cases. At the execution point a reference goes out of scope, we need
to know from which location it was derived, so we can write back the updated
value correctly. Fig.~\ref{fig:MutElim} illustrates the approach for doing
this. A new Move prover internal type |Mut<T>| is introduced which carries the
location from which |T| was derived together with the value. It supports the
following operations:

\begin{itemize}
\item |Mvp::mklocal(value, LOCAL_ID)| creates a new mutation value for a local
  with the given local id.  Local ids are transformation generated constants
  kept opaque here.
\item Similarily, |Mvp::mkglobal(value, TYPE_ID, addr)| creates a new
  mutation for a global with given type and address. Notice that in the
    current Move type system, we would not need to represent the address, since
    there can be only one mutable reference into the entire type (via the
    acquires mechanism). However, we keep this more general here, as the Move
    type system might change.
\item With |r' = Mvp::field(r, FIELD_ID)| a mutation value for a subreference is
  created for the identified field.
\item The value of a mutation is replaced with |r' = Mvp::set(r, v)| and
  retrieved with |v = Mvp::get(r)|.
\item With the predicate |Mvp::is_local(r, LOCAL_ID)| one can test whether |r|
  was derived from the given local, and with |Mvp::is_global(r, TYPE_ID, addr)|
  whether it was derived from the specified global. The predicate
  |Mvp::is_field(r, FIELD_ID)| tests whether it is derived from the given field.
\end{itemize}

\begin{figure}[t!]
  \caption{Elimination of Mutable References}
  \label{fig:MutElim}
  \centering
\begin{MoveBoxNumbered}
  fun increment(x: &mut u64) { *x = *x + 1 }
  fun increment_field(s: &mut S) {
    let r = if (s.f > 0) &mut s.f else &mut s.g;
    increment(r)
  }
  fun caller(p: bool): (S, S) {
    let s1 = S{f:0, g:0}; let s2 = S{f:1, g:1};
    let r = if (p) &mut s1 else &mut s2;
    increment_field(r);
    (s1, s2)
  }
  @\transform@
  fun increment(x: Mut<u64>): Mut<u64> { Mvp::set(x, Mvp::get(x) + 1) }
  fun increment_field(s: Mut<S>): Mut<S> {
    let r = if (s.f > 0) Mvp::field(s.f, S_F) else Mvp::field(s.g, S_G);
    r = increment(r);
    if (Mvp::is_field(r, S_F))
      s = Mvp::set(s, Mvp::get(s)[f = Mvp::get(r)]);
    if (Mvp::is_field(r, S_G))
      s = Mvp::set(s, Mvp::get(s)[g = Mvp::get(r)]);
    s
  }
  fun caller(p: bool): S {
    let s1 = S{f:0, g:0}; let s2 = S{f:1, g:1};
    let r = if (p) Mvp::mklocal(s1, CALLER_s1)
            else Mvp::mklocal(s2, CALLER_s2);
    r = increment_field(r);
    if (Mvp::is_local(r, CALLER_s1))
      s1 = Mvp::get(r); @\label{line:WriteBack}@
    if (Mvp::is_local(r, CALLER_s2))
      s2 = Mvp::get(r);
    (s1, s2)
  }
\end{MoveBoxNumbered}
\end{figure}

\Paragraph{Implementation} The Move Prover has a partial implementation of the
illustrated transformation.  The completeness of this implementation has not yet
been formally investigated, but we believe that it covers all of Move, with the
language's simplification that we do not need to distinguish addresses in global
memory locations.\footnote{\TODO{wrwg}{Need to investigate loops!}} (See
discussion of |Mvp::mkglobal| above.) The transformation also relies on that in
Move there are no recursive data types, so field selection paths are statically
known. While those things can be potentially generalized, we have not yet
investigated this direction.

The transformation constructs a \emph{borrow graph} from the program via a data
flow analysis. The borrow graph tracks both when references are released as well
as how they relate to each other: e.g. |r' = &mut r.f| creates a edge from |r|
to |r'| labelled with |f|, and |r' = &mut r.g| creates another also starting
from |r|. For the matter of this problem, a reference is not released until a
direct or indirect borrow on it goes out of scope; notice that its lifetimes in
terms of borrowing is larger than the scope of its usage. The borrow analysis is
\emph{inter-procedural} requiring computed summaries for the borrow graph of
called functions.

The resulting borrow graph is then used to guide the transformation, inserting
the operations of the |Mut<T>| type as illustrated in
Fig~\ref{fig:MutElim}. Specifically, when the borrow on a reference ends, the
associated mutation value must be written back to its parent mutation or the
original location (e.g. line~\ref{line:WriteBack} in
Fig.~\ref{fig:MutElim}). The presence of multiple possible origins leads to case
distinctions via |Mvp::is_X| predicates; however, these cases are rare in actual
Move programs.


%%% Local Variables:
%%% mode: latex
%%% TeX-master: "main"
%%% End:
